\section{Goals}
\setcounter{subsection}{0}
\renewcommand*{\thesubsection}{\Alph{subsection}.}

We have identified the following project deliverables we aim to achieve by the end of the course.

\paragraph{}
Firstly, we would model the base behaviour of message passing between nodes used in chain replication using TLA+ as a simplified specification. This will act as our minimal viable prototype (MVP) where we do not model an unstable network, reordering or dropping messages between nodes, as well as node failures. We can use the MVP to check liveness conditions and the invariants mentioned in the paper.

\paragraph{}
After an MVP has been produced, we would create an implementation of the chain replication specification in TLA+, adding the behaviour of node failure and recovery. This will be done in stages, modelling failure in the head node, then the tail node and finally the middle nodes. At each stage, we will check if the implementation still follows the specifications mentioned in the MVP. The final implementation, properly modelling all 3 possible failures and recovery actions will act as our main goal and deliverable in this project.

\paragraph{}
Possible stretch goals were also considered, this includes extending our implementation to model additional events such as adding new nodes to the chain when the chain length gets below a specified threshold and modelling possible network errors during message passing such as reordered or dropped requests.

\paragraph{}
