\section{Related Works}
\setcounter{subsection}{0}
\renewcommand*{\thesubsection}{\Alph{subsection}.}

\paragraph{}
The closest comparison is perhaps Primary/Backup protocol. The significant difference regarding their topology is that Primary/Backup protocol uses parallel backup servers where the primary is responsible for all incoming requests from clients and synchronizes the data to all backups. 
% The scenario requires more duration from transient outages since the primary is responsible for everything. 
Chain Replication uses serial servers that traverse the request from one to another. It is then easier to remove a faulty node or add a new node, resulting in lower duration needed for transient outages. However, in Primary/Backup protocol, it requires less time to respond to the clients' requests because the primary is directly connected to the backup - it is proportional to the maximum latency of the working nodes. In Chain Replication, a message must traverse through all the nodes before the client can get its reply. Hence, the response time is proportional to the sum of all the nodes' latency.

\paragraph{}
Another related work is GFS (Google File System)~\cite{ghemawat2003google}. It provides a fault-tolerant service using cheap hardware products. However, it does not consider consistency because the nature of their file usability does not require it to be consistent. Once data are written, they rarely modify them. Any incoming data will be just appended to the file without keeping them being serialized in concurrent events. Chain replication, however, while keeping a fault-tolerant service, it also preserves strong consistency.
