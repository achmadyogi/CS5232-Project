\section{Explanation}
\setcounter{subsection}{0}
\renewcommand*{\thesubsection}{\Alph{subsection}.}

\paragraph{}
Chain Replication is an algorithm for supporting large-scale storage services that exhibit high throughput and availability without sacrificing strong consistency guarantees. Chain replication has real-world uses in its domain, being used in systems such as Delta, a low-dependency object storage system by Meta. This will be an interesting and non-trivial algorithm to formalise in TLA+ as, unlike transaction commit as covered in lectures where resource managers can function independently of one another, the equivalent resource managers (nodes) in Chain Replication interact directly with its successor and predecessor nodes.

\paragraph{}
Furthermore, as nodes follow a strict order, there is no longer symmetry in the algorithm in regard to node failures. Concretely, the failure of the head node would lead to vastly different state transitions than a failure in a middle or tail node. The interaction between multiple node failures will also be very interesting to analyse as a result of this increased complexity.  

\paragraph{}
All these factors combine to give rise to complex and non-trivial state transitions that would be impractical to be analysed without tools like TLA+. By formalising the algorithm in TLA+, we could analyse these complex traces to ensure that the invariants mentioned in the paper hold for all traces, helping prove the correctness of the algorithm in the process. We could check liveness conditions to ensure that no combination of node failures would result in inconsistency between the head and tail nodes. We would also gain better knowledge of the utility and limitations of TLA+ through this process.
