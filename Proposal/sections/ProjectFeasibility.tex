\section{Project Feasibility}
\setcounter{subsection}{0}
\renewcommand*{\thesubsection}{\Alph{subsection}.}

\paragraph{}
We can define correctness based on the safety and liveness of the protocol. The paper demonstrates Chain Replication performance compared with Primary/Backup in terms of throughput, which is a
good indicator of liveness. However, the paper does not demonstrate its safety further, so we focus more on its safety and consistency in this project, which the paper does not include. We might face several challenges when building a formal specification for the protocol during the implementation.

\paragraph{}
First, the fundamental part is that we have to establish the equivalent TLA+ properties and invariance towards the protocol that the paper only explains its high-level overview. We need to apply many state transitions such that we might produce some implicit modifications that give different outputs or behaviors.

\paragraph{}
Second, in Chain Replication, all servers participate in the message deliverability (serial). Hence, some data might be lost if server failures happen too frequently. If we allow message loss or server failure in TLA+, we need to find some way to limit their occurrence such that the probability of the events is less than half or smaller.